\documentclass[10pt]{beamer}

\usetheme{metropolis}
\usepackage{appendixnumberbeamer}

\usepackage[utf8]{inputenc}
\usepackage[spanish]{babel}

\usepackage{booktabs}

\usepackage{pgfplots}
\usepgfplotslibrary{dateplot}

\usepackage{xspace}

\usepackage{amssymb}
\usepackage{graphicx}
\usepackage{amsmath}


\newcounter{saveenumi}
\newcommand{\seti}{\setcounter{saveenumi}{\value{enumi}}}
\newcommand{\conti}{\setcounter{enumi}{\value{saveenumi}}}

\resetcounteronoverlays{saveenumi}


\newcommand{\Cfonta}{\fontsize{6.7}{9.5}\selectfont}
\newcommand{\Cfontb}{\fontsize{5.7}{9.5}\selectfont}
\newcommand{\Cfont}{\fontsize{5.5}{7.2}\selectfont}
\newcommand{\Cfonti}{\fontsize{8.5}{7.2}\selectfont}
\newcommand{\themename}{\textbf{\textsc{metropolis}}\xspace}

\newcommand{\p}{\mathbb{P}}


\begin{document}

\begin{frame}{Enunciado}

    Si $\p\ $ es un procedimiento efectivo cuyo conjunto de datos de entrada es
    $\omega x \Sigma^{*}$, entonces el conjunto $\{(x, \alpha) \in \Sigma^{*}: \p\ termina\ partiendo\ de\ $
    $ (x, \alpha)\}$ es $\Sigma$-efectivamente enumerable.


\end{frame}

\begin{frame}{Planteo}
  \begin{itemize}[<+->]
    \item Supongamos $\p\ $ un procedimiento efectivo cuyo conjunto de datos de
    entrada es $\omega x \Sigma^{*}$
    \item Probaremos entonces, que el conjunto $S = \{(x, \alpha) \in \Sigma^{*}: \p\ termina\ partiendo\ de\ $
    $ (x, \alpha)\}$ es $\Sigma$-efectivamente enumerable
  \end{itemize}

\end{frame}



\begin{frame}{Prueba: dos casos}
  \begin{itemize}[<+->]
    \item Si $S = \emptyset$: $S$ es $\Sigma$-efectivamente enumerable por definición
    \item Si $S \neq \emptyset$: daremos un procedimiento efectivo, $\p_{s}\ $
    que enumere a $S$
  \end{itemize}

\end{frame}


\begin{frame}{Prueba: caso $S \neq \emptyset$}
  Por definición, como $\p_{s}\ $ debe enumerar a $S$, entonces debe cumplir que:

  \begin{enumerate}[<+->]
    \item El conjunto de entrada de $\p_{s}\ $ es $\omega$
    \item $\p_{s}\ $ siempre termina
    \item El conjunto de salida de $\p_{s}\ $ es $S$
  \end{enumerate}

\end{frame}



\begin{frame}{Prueba: caso $S \neq \emptyset$}
  Daremos explicitamente el procedimiento $\p_{s}\ $
  \begin{itemize}[<+->]
    \item Como $S \neq \emptyset$ entonces existe, al menos, un elemento en $S$.
    Sea $(x_{1}, \alpha_{1})$ un elemento cualquiera de $S$
    \item Supongamos $<$ un orden sobre $\Sigma^{*}$
  \end{itemize}

\end{frame}



\begin{frame}{Prueba: caso $S \neq \emptyset$}
  \underline{$\p_{s}\ $}: Toma como entrada un valor $x \in \omega$
  \begin{itemize}[<+->]
    \item[ ] \underline{Etapa 1}: Si el valor de entrada, $x$, es igual a $0$
    y devolver $(x_{1}, \alpha_{1})$ y detenerse. \\
    Si no, correr $\p\ $ una cantidad $(x)_{1}$ pasos con entrada
    $(\ (x)_{2},\ *^{<}(\ (x)_{3})\ )$. Si luego de una cantidad $(x)_{1}$
    de pasos, $\p\ $ termina, devolver $(\ (x)_{2},\ *^{<}(\ (x)_{3})\ )$ y
    detenerse. Si no termina, devolver $(x_{1}, \alpha_{1})$ y detenerse.

  \end{itemize}
\end{frame}

\begin{frame}{Prueba: caso $S \neq \emptyset$}
  Dado que $\p_{s}\ $ es claramente efectivo, veamos que cumple con 1, 2 y 3
  \begin{itemize}[<+->]
    \item 1 y 2 se cumplen trivialmente. Veamos 3:
    \begin{itemize}[<+->]
      \item Si $(x, \alpha)$ es un elemento cualquiera de $S$, por definición,
      $\p\ $ termina partiendo de $(x, \alpha)$.
      \item Sea $p$ la cantidad de pasos que necesita $\p\ $ para terminar partiendo
      de $(x, \alpha)$
      \item Entonces, para el valor de entrada
      $z = 2^{p+1} * 3^{x} * 5^{\#^{<}(\alpha)}$ $\p_{s}\ $ termina y da como salida
      $(x, \alpha)$.
    \end{itemize}
    \item Luego, se cumple 1, 2 y 3
  \end{itemize}

\end{frame}


\begin{frame}{Prueba: caso $S \neq \emptyset$}
  Como dimos un procedimiento efectivo, $\p_{s}\ $ que cumple con 1, 2 y 3, este
  enumera a S. \\
  Por lo tanto, S es efectivamente enumerable
\end{frame}

\end{document}
